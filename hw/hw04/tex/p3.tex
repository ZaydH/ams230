\begin{problem}\label{prob:03}%%
  Code Algorithm 7.5, and test it on the extended Rosenbrock function
  
  \[ f(x) = \sum_{i=1}^{n/2} \left[ \alpha(x_{2i} - x^2_{2i-1})^2 + (1-x_{2i-1})^2 \right] \text{.} \]
  
  where $\alpha$ is a parameter that you can vary (for example, 1 or 100). The solution is $\xopt=(1,1,...,1)\transpose$, $f^{*}=0$. Choose the starting point as $(-1,-1,...,-1)\transpose$. Observe the behavior of your program for various values of the memory parameter $m$.
\end{problem}

For even valued $n$, the gradient of $f$ is:

\[
\nabla f(x) = \left\{
                \begin{array}{cl}
                  -4\alpha x_{j}(x_{j+1} - x^2_{j}) - 2x_{j}(1-x_{j}) & j \text{ is odd}\\
                  2\alpha(x_j - x^{2}_{j-1}) & j \text{ is even}
                \end{array}
              \right.
\]

\noindent
where ${j \in \{1,\ldots,n\}}$.  Since LBFGS relies on line search, $\phi(\alpha) = f(x_k + \alpha_k p_k)$.  In addition, by the chain rule, 

\begin{aligncustom}
  \phi'(\alpha) &= \frac{\partial f(x_k+\alpha p_k)}{\partial \alpha} \\
                &= \frac{\partial f(x_k+\alpha p_k)}{\partial x_k} \cdot \frac{\partial x_k}{\partial \alpha} \\
                &= \nabla f(x_k + \alpha p_k) \cdot p_k \text{.}
\end{aligncustom}

Table~\ref{tab:p03:ExperimentParameters} lists the experiment parameters for this problem.  The implementation of line search remains essentially unchanged from that used in homeworks~\#1 and~\#2.  When $\alpha$ in the extended Rosenbrock function was set to~1, the model converged too quickly.  As such, $\alpha$ was set to~100.

\begin{table}[h]
  \centering
  \caption{Experiment parameters for problem~\#3}\label{tab:p03:ExperimentParameters}
  \begin{tabular}{|c|c|}
    \hline
    \textbf{Name} & \textbf{Value} \\\hline\hline
    $n$         & 1,000 \\\hline
    $\alpha$    & 100 \\\hline
    $x_0$       & $[-1]^{n}$ \\\hline
    $\alpha_0$  & 0 \\\hline
    $c_1$       & 0.1 \\\hline
    $c_2$       & 0.45 \\\hline
  \end{tabular}
\end{table}

When $m=0$, i.e., the algorithm behaves in a memoryless fashion and converges as traditional line search, it took 2,340 iterations to converge (not shown).  Figure~\ref{fig:p03:ResultsPlot} shows the results for different values of $m$.  As expected, when $m=1$, it took the longest to converge.  For $m\geq 2$, the convergence rate was essentially the same.  While for $m=2$ and $m=5$, the convergence was marginally, the difference was marginal and not unexpected as excluding some gradients may cause the algorithm to perform better in some cases.  For $m=10$, $m=20$, and maximum $m$ (i.e., save all previous results), the algorithm performed essentially the same.  This means that only the most recent gradients affected the results and memory of distant iterations does not improve the results for this function.

\begin{figure}
 \centering
%\tikzstyle{dotted}=                  [dash pattern=on \pgflinewidth off 2pt]
%\tikzstyle{densely dotted}=          [dash pattern=on \pgflinewidth off 1pt]
%\tikzstyle{loosely dotted}=          [dash pattern=on \pgflinewidth off 4pt]
%\tikzstyle{dashed}=                  [dash pattern=on 3pt off 3pt]
%\tikzstyle{densely dashed}=          [dash pattern=on 3pt off 2pt]
%\tikzstyle{loosely dashed}=          [dash pattern=on 3pt off 6pt]
%\tikzstyle{dashdotted}=              [dash pattern=on 3pt off 2pt on \the\pgflinewidth off 2pt]
%\tikzstyle{densely dashdotted}=      [dash pattern=on 3pt off 1pt on \the\pgflinewidth off 1pt]
%\tikzstyle{loosely dashdotted}=      [dash pattern=on 3pt off 4pt on \the\pgflinewidth off 4pt]
\begin{tikzpicture}[
    % Dashed style is shown here: https://tex.stackexchange.com/questions/45275/tikz-get-values-for-predefined-dash-patterns
    dashdotted/.style         = {dash pattern=on 3pt off 2pt on \the\pgflinewidth off 2pt},
    loosely dashed/.style     = {dash pattern=on 3pt off 6pt},
    densely dashdotted/.style = {dash pattern=on 3pt off 1pt on \the\pgflinewidth off 1pt},
    densely dotted/.style     = {dash pattern=on \pgflinewidth off 1pt},
  ]
  \pgfplotstableread[col sep=comma] {pgfplots/plot_data/lbfgs_m=1.csv}\mOne
  \pgfplotstableread[col sep=comma] {pgfplots/plot_data/lbfgs_m=2.csv}\mTwo
  \pgfplotstableread[col sep=comma] {pgfplots/plot_data/lbfgs_m=5.csv}\mFive
  \pgfplotstableread[col sep=comma] {pgfplots/plot_data/lbfgs_m=10.csv}\mTen
  \pgfplotstableread[col sep=comma] {pgfplots/plot_data/lbfgs_m=20.csv}\mTwenty
  \pgfplotstableread[col sep=comma] {pgfplots/plot_data/lbfgs_max_mem.csv}\maxMem
  \begin{axis}[
      width=10cm,
      height=10cm,
      xmin=0,
      xmax=35,
      minor x tick num=3,
      ymin=-14,
      ymax=6,
      minor y tick num=1,
      every tick label/.append style={font=\scriptsize},  % Reduce axis font size
      xlabel={Iteration \#},
      ylabel={$\log_{10} f(x)$},
      xlabel style={font=\scriptsize},
      ylabel style={font=\scriptsize},
      ylabel shift = -4pt,
      title={L-BFGS performance comparison for different values of $m$},
    ]
    \addplot[
      color=red,
      thick
    ]
    table[x index=0,y index=1] {\mOne};
    \addplot[
      color=green,
      dashdotted,
      thick
    ]
    table[x index=0,y index=1] {\mTwo};
    \addplot[
      color=orange,
      loosely dashed,
      thick
    ]
    table[x index=0,y index=1] {\mFive};
    \addplot[
      color=blue,
      densely dashdotted,
      thick
    ]
    table[x index=0,y index=1] {\mTen};
    \addplot[
      color=purple,
      densely dotted,
      thick
    ]
    table[x index=0,y index=1] {\mTwenty};
    \addplot[
      color=black,
      dashed,
      thick
    ]
    table[x index=0,y index=1] {\maxMem};
    \legend{$m=1$, $m=2$, $m=5$, $m=10$, $m=20$, Max $m$};
  \end{axis}
\end{tikzpicture}

  \caption{Convergence of L-BFGS for different values of $m$ on the extended Rosenbrock function with $\alpha=100$}\label{fig:p03:ResultsPlot}
\end{figure}
