\begin{problem}\label{prob:01}%%
  Consider the dog-leg path:
  
  \[
    \tilde{p}(\tau) = \left\{
                        \begin{array}{cl}
                          \tau p^{U}, & 0\leq \tau \leq 1\\
                          p^{U} + (\tau - 1)(p^{B} - p^{U}), & 1 \leq \tau \leq 2
                        \end{array}
                      \right.
  \]
  
  \noindent
  where 
  
  \begin{aligncustom}
    p^{U} &= -\frac{\norm{g}^2}{g\transpose B g} g\\
    B \cdot p^{B} &= -g \text{.}
  \end{aligncustom}

  \noindent
  Support that symmetric matrix $B$ and vector $g$ satisfy:
  
  \begin{enumerate}
    \item $g\transpose B g > 0$
    \item $\left(p^{U}\right)\transpose \left(p^{B} - p^{U}\right) > 0$
  \end{enumerate}

  \noindent
  Prove that:

  \begin{enumerate}[i.]
    \item\label{item:p01:prop1} $\norm{\tilde{p}(\tau)}$ is an increasing function of $\tau$
    \item\label{item:p01:prop2} $m(\tilde{p}(\tau))$ is a decreasing function of $\tau$
  \end{enumerate}

  \noindent
  where $m(p) = g\transpose p + \frac{1}{2}p\transpose B p$.
\end{problem}

\begin{proof}
  Based off Lemma~4.2 from Nocedal and Wright.  There are two cases based on the value of $\tau$.
  
  \noindent
  \textit{Case~\#1}: $0 \leq \tau \leq 1$
  
  \noindent
  \textbf{Property~\ref{item:p01:prop1}}: $\norm{\tilde{p}(\tau)} = \tau \norm{p^U}$.  This is clearly increasing for $\tau \in [0,1]$ since $\norm{p^U}$ is strictly positive for $p^{U} \ne 0$.
  
  \noindent
  \textbf{Property~\ref{item:p01:prop2}}: This can be simplified via:
  
  \begin{aligncustom}
    m(\tilde{p}(\tau)) &= \tau g\transpose p^{U} + \frac{\tau^{2}}{2} \left(p^{U}\right)\transpose B p^{U} \\
    &= -\tau \frac{\left(\norm{g}^2\right)^{2}}{g\transpose B g} + \frac{\tau^{2}}{2}\frac{\left(\norm{g}^2\right)^{2}}{\left(g\transpose B g\right)^{2}}g\transpose B g  \\
    &= \left(-\tau + \frac{\tau^2}{2}\right)\frac{\left(\norm{g}^2\right)^{2}}{g\transpose B g}
  \end{aligncustom}

  \noindent
  Both $\norm{g}^{2}$ and $g\transpose B g$ are strictly positive making their ratio also strictly positive.  The ratio is also fixed with respect to $\tau$ meaning it can be treated as a positive constant.  For $\tau \in [0,1]$, $-\tau + \tau^2/2$ is strictly decreasing.

  \noindent
  \textit{Case~\#2}: $1 \leq \tau \leq 2$\\
  
  \noindent
  \textbf{Property~\ref{item:p01:prop1}}: Define $h(\alpha)$ where $\alpha \in (0,1)$ as:
  
  \begin{aligncustom}
    h(\alpha) &= \frac{1}{2}\norm{\tilde{p}(1+\alpha)}^2\\
              &= \frac{1}{2} \norm{p^{U} +\alpha(p^{B}-p^{U})} \\
              &= \frac{1}{2}\norm{p^U}^2 + \alpha\left(p^{U}\right)\transpose(p^{B}-p^{U}) + \frac{1}{2}\alpha^2\norm{p^{B}-p^{U}}^2
  \end{aligncustom}

  \noindent
  To prove the property, it is sufficient to show that $h'(\alpha) \geq 0$ for $\alpha \in (0,1)$.  Therefore,
  
  \begin{aligncustom}
    h'(\alpha)  &= -\left(p^{U}\right)\transpose \left(p^{U} - p^{B} \right) + \alpha\norm{p^{U} - p^{B}}^{2}\\
                &\geq -(p^{U})\transpose (p^{U}-p^{B}) \\
                &= \left(\frac{g\transpose g}{g\transpose B g}\right)g\transpose\left(-\left(\frac{g\transpose g}{g\transpose B g}\right)g + B^{-1}g \right) \\
                &= g\transpose g \frac{g\transpose B^{-1} g}{g\transpose B g}\left[ 1 - \frac{(g\transpose g)^{2}}{\left(g\transpose B g\right) \left(g\transpose B^{-1} g \right)  } \right] \\
                &= g\transpose g \frac{g\transpose B^{-1} g}{g\transpose B g}\left[ 1 - \frac{\norm{g}^{4}}{\left(g\transpose B g\right) \left(g\transpose B^{-1} g \right)  } \right] \text{.}
%                &\geq 0
  \end{aligncustom}

  \noindent
  If $B$ is positive definite, then so is $B^{-1}$.  Therefore, 
  
  \[g\transpose g \frac{g\transpose B^{-1} g}{g\transpose B g} > 0\text{,}\]
  
  \noindent
  and can be treated as a positive constant with respect to $\alpha$.  By the Cauchy-Schwarz inequality, $g\transpose B g \leq \norm{g} \norm{B} \norm{g}$.  This simplifies term:
  
  \[ \frac{\norm{g}^{4}}{\left(g\transpose B g\right) \left(g\transpose B^{-1} g \right)  } \geq \frac{1}{\norm{B}\norm{B^{-1}}} \textrm{.}\]
  
  \noindent
  Again by Cauchy-Schwarz, $\norm{B}\norm{B^{-1}} \geq \norm{BB^ {-1}} = \norm{I} = 1$.  Therefore, 
  
  \[ \frac{1}{\norm{B}\norm{B^{-1}}} \leq 1 \textrm{.}\]
 
  \noindent
  \textbf{Property~\ref{item:p01:prop2}}: Define  $\hat{h}(\alpha)=m(\tilde{p}(1+\alpha))$.  If $h'(\alpha) \leq 0$ for $\alpha \in (0,1)$ then the property holds.  Using the definition of $\tilde{p}(\tau)$ in the exercise description and the definition of trust region, we find:
  
  \begin{aligncustom}
    \hat{h}'(\alpha)  &= (p^{B} - p^{U})\transpose (g + Bp^{U}) \\
                      &\leq (p^{B} - p^{U})\transpose (g+Bp^{U} + B(p^{B} - p^{U})) \\
                      &= (p^{B} - p^{U})\transpose (g + Bp^{B}) = 0
  \end{aligncustom}

  \noindent
  given $B \cdot p^{B} = -g$.
\end{proof}
