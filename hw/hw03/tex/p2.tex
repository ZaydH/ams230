\begin{problem}\label{prob:02}%%
  Code Algorithm~4.1 in Nocedal and Wrihght with:
  
  \begin{enumerate}
    \item Cauchy point method for the subproblem
    \item Dog-leg method based on the results for Exercise~\ref{prob:01}.
  \end{enumerate}

  \noindent
  Test and compare the performance of the methods on the following problem:
  
  \[\min_{x\in \mathbb{R}^n} f(x) = \log\left(1+x\transpose Qx\right)\]
  
  \noindent
  where $Q$ is a symmetric and positive definite matrix.
\end{problem}


$Q$~was constructed in the same way as homework~\#2 using QR~decomposition and a diagonal matrix of eigenvalues. The gradient,~$g$, of $f$ is defined as: 

\[g = \nabla f(x) = \frac{2Qx}{1+x\transpose Q x}\text{.}\]

\noindent
The Hessian can then be found via the quotient rule as shown below.

\begin{aligncustom}
  \nabla f^{2} (x)  &= \frac{d}{dx} g = \frac{d}{dx} \frac{2Qx}{1+x\transpose Q x}\\
                    &= \frac{1}{1+x\transpose Q x} \frac{d}{dx} \left(2Qx \right) + 2Qx \frac{d}{dx} \left(\frac{1}{1+x\transpose Q x} \right) \\
                    &= \frac{2Q}{1+x\transpose Q x} - 2Qx \left(\frac{(2Qx)\transpose}{(1+x\transpose Q x)^{2}} \right)
\end{aligncustom}

\noindent
For most $x$, the Hessian was not positive definite.  To approximate the Hessian, $B_k$ was set to $Q$.  While this is not exactly the Hessian, it achieved good convergence.  I tried an alternate approach where $B_k=Q$ \textit{only} if the Hessian was not positive definite.  However, this was actually slower to converge for both Cauchy Points and Dogleg.  As such, those results are not reported in this document.

Table~\ref{tab:p02:experimentParameters} lists the parameters used in the experiments.   Note that $\mathcal{U}(a,b)$ represents a uniform random variable selected from the range $[a,b)$. Since $Q$ is positive definite, then for all $x \ne [0]^n \implies x\transpose Q x > 0$.  Therefore, since $\log$ is monotonically increasing, $f$ is minimized when $\xopt = [0]^n$.

\begin{table}[h]
  \caption{Parameters used in the experiments for Exercise~\#\ref{prob:02}}\label{tab:p02:experimentParameters}
  \centering
  \begin{tabular}{|c|c|}
    \hline
    \textbf{Name} & \textbf{Value} \\\hline
    \hline
    $\log$      &   $\log_{10}$\\\hline
    $n$         &   100\\\hline
    $\lambda$   &   $\mathcal{U}(10,1000)$ \\\hline
    $B$         &   $Q$ \\\hline
    $\Delta_0$  &   1 \\\hline
    $\hat{\Delta}$& 10,000 \\\hline 
    $\eta$      &   0.1\\\hline
    $x_0$       &   Random vector from $[0,1)^{n}$\\\hline
    $\xopt$     &   $[0]^n$\\\hline
  \end{tabular}
\end{table}

Figure~\ref{fig:p02:ResultsComparison} shows the performance of the Cauchy Point and Dogleg methods for the same $x_0$.  Observe that Cauchy Point took longer to converge than Dogleg.  The exact number of steps to converge varied by upto 75\% depending on the value of the random $x_0$ and $Q$.  In addition, the difference in the converge rate of Cauchy Point and Dogleg varied with $x_0$ and $Q$.

\begin{figure}
  \centering
\begin{tikzpicture}
  \pgfplotstableread[col sep=comma] {pgfplots/plot_data/cauchy.csv}\cauchy
  \pgfplotstableread[col sep=comma] {pgfplots/plot_data/dog_leg.csv}\dogleg
  \begin{axis}[
      width=10cm,
      height=10cm,
      xmin=0,
      xmax=4600,
      minor x tick num=3,
      ymin=-16,
      ymax=2,
      minor y tick num=4,
      every tick label/.append style={font=\scriptsize},  % Reduce axis font size
      xlabel={Iteration \#},
      ylabel={$\log_{10} f(x)$},
      xlabel style={font=\scriptsize},
      ylabel style={font=\scriptsize},
      ylabel shift = -4pt,
      title={Comparison of the performance of Cauchy Points and Dogleg methods},
      legend pos={south west},
    ]
    \addplot[
      color=blue,
      thick
    ]
    table[x index=0,y index=1] {\cauchy};
    \addplot[
      color=red,
      dashed,
      thick
    ]
    table[x index=0,y index=1] {\dogleg};
    \legend{Cauchy Points, Dogleg};
  \end{axis}
\end{tikzpicture}

  \caption{Comparison of the performance of Cauchy Point and Dog Leg}\label{fig:p02:ResultsComparison}
\end{figure}

