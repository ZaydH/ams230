\begin{problem}
  Exercise 2.1 in Nocedal and Wright.

  Compute the gradient, $\nabla f(x)$, and Hessian, $\nabla^{2}f(x)$ of the Rosenbrock function

  \[ f(x) = 100(x_{2} - x_{1}^{2})^{2} + (1-x_{1})^{2} \textrm{.} \]
\end{problem}

\noindent
The gradient $\nabla f(x)$ is:

\begin{aligncustom}
  \nabla f(x) &=  \begin{bmatrix}
                    \frac{\partial f(x)}{\partial x_1} \\
                    \frac{\partial f(x)}{\partial x_2}
                  \end{bmatrix} \\
              &=  \begin{bmatrix}
                    -400 x_{1} (x_{2} - x_{1}^{2}) - 2(1-x_{1}) \\
                    200 (x_{2} - x_{1}^{2})
                  \end{bmatrix}\\
              &=  \boxed{
                    \begin{bmatrix}
                      400 x_{1}^3 - 400x_{1}x_{2} +2x_1 -2 \\
                      200 x_{2} - 200 x_{1}^{2}
                    \end{bmatrix}
                  } \textrm{.} 
\end{aligncustom}

\noindent
The Hessian~$\nabla^2 f(x)$ equals:                

\begin{aligncustom}
  \nabla^{2} f(x) &=  \begin{bmatrix}
                        \frac{\partial^{2} f(x)}{\partial x_1^{2}} & \frac{\partial^{2} f(x)}{\partial x_1 \partial x_2} \\
                        \frac{\partial^{2} f(x)}{\partial x_2 \partial x_1} & \frac{\partial^{2} f(x)}{\partial x_{2}^{2}}
                      \end{bmatrix} \\
                &=  \boxed{
                      \begin{bmatrix}
                        1200x_{1}^{2} -400x_2 + 2 & -400 x_{1} \\
                        -400x_{1} & 200
                      \end{bmatrix}
                    }
\end{aligncustom}



\begin{subproblem}
  Show that $x^{*} = (1,1)\transpose$ is the only local minimizer of the function, and that the Hessian at that point is positive definite.
\end{subproblem}

To determine the minimizer(s) (if any), set the gradient equal to~0.  Therefore,

\[  \begin{bmatrix}
      400 x_{1}^3 - 400x_{1}x_{2} +2x_1 -2 \\
      200 x_{2} - 200 x_{1}^{2}
    \end{bmatrix}  
    = \begin{bmatrix}
        0 \\ 0
      \end{bmatrix} \textrm{.} \]

\noindent
Therefore, $x_2 = x_{1}^{2}$.  If we substitute this into the first equation, we get:

\begin{aligncustom}
  400 x_{1}^3 - 400x_{1}x_{2} +2x_1 -2 &= 0 \\
  400 x_{1}^3 - 400x_{1}x_{1}^2 +2x_1 -2 &= 0 \\
  2x_1 -2 &= 0 \\
  x_1 = 1
\end{aligncustom}

\noindent
It is clear then that the only root is $\boxed{(1,1)\transpose}$.

The Hessian matrix~$\nabla^{2}f(x^{*})$ of the Rosenbrock function equals:

\[
\nabla^{2}f(x^{*}) =  \begin{bmatrix}
                        802 & -400 \\
                        -400 & 200
                      \end{bmatrix} \textrm{.}\]

\noindent
Hessian matrix,~$\nabla^{2}f(x^{*})$, is positive definite (PD) if all of its eigenvalues are positive.  These are found via:

\begin{aligncustom}
  \abs{A-\lambda I}  =  \begin{vmatrix}
                          802 - \lambda  & -400 \\
                          -400           & 200 - \lambda
                        \end{vmatrix}
                    &=  0 \\
  160400 - 1002 \lambda + \lambda^2 - 160000 &= 0 \\
  400- 1002 \lambda + \lambda^2 &= 0
\end{aligncustom}

\noindent
Solving for $\lambda$ (in Matlab), the eigenvalues are $\boxed{\lambda_1\approx 0.3994}$ and $\boxed{\lambda_2 \approx 1001.6}$.  Therefore, $\nabla^{2}f(x^{*})$ is PD.
