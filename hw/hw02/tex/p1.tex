\begin{problem}
  Exercise 5.2 in Nocedal and Wright.

  Show that if the nonzero vectors $p_0,p_1,\ldots,p_l$ satisfy Eq.~(5.5) from Nocedal and Wright that
  
  \[p_i\transpose A p_{j} = 0, \textnormal{ for all } i \ne j\text{,}\]
  
  \noindent
  where $A$ is symmetric and positive definite, then these vectors are linearly independent. (This result implies that $A$ has at most $n$ conjugate directions.)
\end{problem}

\begin{proof} 
  By contradiction.  
  
  Assume vectors $\{p_0,p_1,\ldots,p_l\}$ are $A$-orthogonal but are not linearly independent.  That means there is a set constants $\{\alpha_0,\alpha_1,\ldots,\alpha_{l}\}$ not all equal to zero such that:
  
  \[ \alpha_{0}p_{0} + \alpha_{1}p_{1} + \cdots +\alpha_{l}p_{l} = \vec{0} \text{.}\]
  
  \noindent
  For some $0\leq i \leq l$ where $\alpha_i> 0$, multiply both sides by $Ap_i$ yielding:
  
  \[ \alpha_{0}p_{0}\transpose Ap_{i} + \alpha_{1}p_{1}\transpose Ap_{i} + \cdots +\alpha_{l}p_{l}\transpose Ap_{i} = \vec{0}\transpose Ap_{i} \text{.}\]
  
  \noindent
  Since $p_{i}$ and $p_{j}$ are $A$-orthogonal for ${i\ne j}$, this simplifies to:
  
  \[ \alpha_{i}p_{i}\transpose Ap_{i} = 0 \text{.}\]
  
  \noindent 
  This is a contradiction since ${\alpha_i \ne 0}$ and $A$ is symmetric, positive definite, i.e.,~${x\transpose A x > 0}$ for all non-zero~$x$.
\end{proof}
