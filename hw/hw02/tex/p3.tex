\begin{problem}
  In this problem we will test the performance of linear conjugate gradient method for minimizing quadratic function $f(x) = \frac{1}{2}x\transpose Ax - b \transpose x$, where $A$ is symmetric and positive definite.

  \begin{itemize}
    \item Program CG Algorithm~5.2 in the textbook.

    \item Apply your program on a symmetric and positive definite matrix $A$ of dimension $10^{3} \times 10^{3}$ with eigenvalues uniformly distributed between~$10$ and~$10^3$. Test the convergence of the algorithm and compare your numerical findings with the theoretical result shown in formula (5.36) in the textbook.

    \item Change the distribution of eigenvalues of $A$ so that some eigenvalues are distributed between 9~and~11, the rest of eigenvalues are distributed between 999 and 1001, i.e.,~two clusters around 10~and~1000 with radius~1. Test the convergence performance on such matrix. Extra points if you can explain your numerical findings using the theoretical convergence results discussed in the lecture.
  \end{itemize}
\end{problem}

The Python source code for the implementation of the linear conjugate gradient method as detailed in Algorithm~5.2 is included at the end of this submission.  The specific parameters used are enumerated in Table~\ref{tab:p03:ExperimentParams}.  Distribution $\mathcal{U}(a,b)$ indicates a uniform random variable drawn from the range $[a,b)$.  Note that for all experiments in this program problem, the same random $x_0$ vector was used to ensure consistency.

\begin{table}[h]
  \centering
  \caption{Parameters for the linear conjugate gradient experiments of problem~\#3}
  \begin{tabular}{|c|c|}
    \hline
    \textbf{Parameter} & \textbf{Value} \\
    \hline\hline
    $n$ & 1,000 \\ \hline
    $x_0$ & $2\cdot\text{rand}(n,1)$ \\ \hline
    Uniform & $\mathcal{U}(10,1000)$ \\ \hline
    Bimodal & 10\% from $\mathcal{U}(9,10)$ and 90\% from $\mathcal{U}(999,1001)$\\ \hline
  \end{tabular}
\end{table}

Figures~\ref{fig:p03:uniformEigenvalues} and~\ref{fig:p03:bimodalEigenvalues} show the performance of the linear conjugate gradient with $n$ uniform and $n$ bimodal eigenvalues respectively.  Similar to the results shown in Figure~5.4 of Nocedal and Wright, matrix~$A$ with uniform eigenvalues takes longer to converge.  For matrix~$A$ with bimodal eigenvalues, it took about 15~steps to fully converge.  Considering that there were approximately 10~eigenvalues from the distribution $\mathcal{U}(9,10)$, these findings align with the theoretical explanation in Nocedal and Wright.

Figures~\ref{fig:p03:bimodalWeightedErr} and~\ref{fig:p03:uniformWeightedErr} show the corresponding weighted error versus the calculated upper bound for the uniform and bimodal eigenvalues respectively.  These results are based on Eq.~(5.36) which states that:

\[ \wnorm{x_k - \xopt} \leq 2\left(\frac{\sqrt{\kappa(A)}-1}{\sqrt{\kappa(A)}+1}\right)\wnorm{x_0 - \xopt} \]

\noindent
Observe that the upper bound tracks well for the uniform eigenvalues.  The upper bound is quite loose when the eigenvalues are clustered bimodally.

\newpage
\begin{figure}[p]
  \centering
  \begin{subfigure}[t]{0.45\textwidth}
    \centering
    \begin{tikzpicture}
  \pgfplotstableread[col sep=comma] {pgfplots/plot_data/p03_uniform_eigen.csv}\thedata
  \begin{axis}[
      width=\plotDim,
      height=\plotDim,
      xmin=0,
      xmax=105,
      minor x tick num=1,
      ymin=-15,
      ymax=6,
      minor y tick num=4,
      every tick label/.append style={font=\scriptsize},  % Reduce axis font size
      xlabel={Iteration \#},
      ylabel={$\log_{10} (f(x) - f(\xopt))$},
      xlabel style={font=\scriptsize},
      ylabel style={font=\scriptsize},
      ylabel shift = -8pt,
      title={Uniform Eigenvalues},
    ]
    \addplot[
%      mark size=0.8pt,
      color=blue,
%      mark=*,
      mark options={blue,fill=blue}
    ]
    table[x index=0,y index=1] {\thedata};
  \end{axis}
\end{tikzpicture}

    \caption{\scriptsize 1,000 $\lambda$ from the distribution $\mathcal{U}(10,10^{3})$}\label{fig:p03:uniformEigenvalues}
  \end{subfigure}
  ~
  \begin{subfigure}[t]{0.45\textwidth}
    \centering
    \begin{tikzpicture}
  \pgfplotstableread[col sep=comma] {pgfplots/plot_data/p03_bimodal_eigen.csv}\thedata
  \begin{axis}[
      width=\plotDim,
      height=\plotDim,
      xmin=0,
      xmax=15,
      minor x tick num=4,
      ymin=-15,
      ymax=6,
      minor y tick num=4,
      every tick label/.append style={font=\scriptsize},  % Reduce axis font size
      xlabel={Iteration \#},
      ylabel={$\log_{10} f(x)$},
      xlabel style={font=\scriptsize},
      ylabel style={font=\scriptsize},
      ylabel shift = -4pt,
      title={Bimodal Eigenvalues},
    ]
    \addplot[
%      mark size=0.8pt,
      color=blue,
%      mark=*,
      mark options={blue,fill=blue}
    ]
    table[x index=0,y index=1] {\thedata};
  \end{axis}
\end{tikzpicture}

    \caption{\scriptsize 1,000 $\lambda$ from distributions $\mathcal{U}(9,11)$ and $\mathcal{U}(999,1001)$ }\label{fig:p03:bimodalEigenvalues}
  \end{subfigure}
  \caption{Cost function $f$ for problem~\#3}
\end{figure}

\begin{figure}[p]
  \centering
  \begin{subfigure}[t]{0.45\textwidth}
    \centering
    \begin{tikzpicture}
  \pgfplotstableread[col sep=comma] {pgfplots/plot_data/p03_uniform_weighted.csv}\thedata
  \begin{axis}[
      width=\plotDim,
      height=\plotDim,
      xmin=0,
      xmax=105,
      minor x tick num=1,
      ymin=-11,
      ymax=5,
      minor y tick num=4,
      every tick label/.append style={font=\scriptsize},  % Reduce axis font size
      legend pos={south west},
      legend style={font=\scriptsize},
      xlabel={Iteration \#},
      ylabel={$\log_{10} \textrm{Weighted Error}$},
      xlabel style={font=\scriptsize},
      ylabel style={font=\scriptsize},
      ylabel shift = -8pt,
      title={Uniform Eigenvalues},
    ]
    \addplot[
%      mark size=0.8pt,
      color=blue,
%      mark=*,
      mark options={blue,fill=blue}
    ]
    table[x index=0,y index=1] {\thedata};
    \addplot[
%      mark size=0.8pt,
      color=red,
      dashed,
%      mark=*,
      mark options={red,fill=blue}
    ]
    table[x index=0,y index=2] {\thedata};
    \legend{Weighted Error, Upper Bound}
  \end{axis}
\end{tikzpicture}

    \caption{\scriptsize Weighted error for 1,000 uniform eigenvalues}\label{fig:p03:uniformWeightedErr}
  \end{subfigure}
  ~
  \begin{subfigure}[t]{0.45\textwidth}
    \centering
    \begin{tikzpicture}
  \pgfplotstableread[col sep=comma] {pgfplots/plot_data/p03_bimodal_weighted.csv}\thedata
  \begin{axis}[
      width=\plotDim,
      height=\plotDim,
      xmin=0,
      xmax=20,
      minor x tick num=1,
      ymin=-11,
      ymax=5,
      minor y tick num=4,
      legend pos={south west},
      legend style={font=\scriptsize},
      every tick label/.append style={font=\scriptsize},  % Reduce axis font size
      xlabel={Iteration \#},
      ylabel={$\log_{10} \textrm{Weighted Error}$},
      xlabel style={font=\scriptsize},
      ylabel style={font=\scriptsize},
      ylabel shift = -8pt,
      title={Bimodal Eigenvalues},
    ]
    \addplot[
%      mark size=0.8pt,
      color=blue,
%      mark=*,
      mark options={blue,fill=blue}
    ]
    table[x index=0,y index=1] {\thedata};
    \addplot[
%      mark size=0.8pt,
      color=red,
      dashed,
      thick,
%      mark=*,
      mark options={red,fill=blue}
    ]
    table[x index=0,y index=2] {\thedata};
    \legend{Weighted Error, Upper Bound} 
  \end{axis}
\end{tikzpicture}

    \caption{\scriptsize Weighted error for 1,000 bimodal eigenvalues}\label{fig:p03:bimodalWeightedErr}
  \end{subfigure}
  \caption{Weighted error results for problem~\#3}
\end{figure}
\clearpage