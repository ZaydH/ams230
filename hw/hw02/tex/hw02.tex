\documentclass{report}


\newcommand{\name}{Zayd Hammoudeh}
\newcommand{\course}{AMS230}
\newcommand{\assnName}{Homework~\#2}

\usepackage{fullpage}
\usepackage[skip=4pt]{caption}      % ``skip'' sets the spacing between the figure and the caption.
\usepackage{tikz}
\usepackage{pgfplots}     % Needed for plotting
\pgfplotsset{compat=newest}
\usepackage{subcaption}

\usepackage{amsmath}      % Allows for piecewise functions using the ``cases'' construct
\usepackage{amsthm}
\usepackage{siunitx}      % Allows for ``S'' alignment in table to align by decimal point
\usepackage{enumitem}     % Reference enumerated item lists.

\usepackage[obeyspaces,spaces]{url} % Used for typesetting with the ``path'' command
\usepackage[hidelinks]{hyperref}    % Make the cross references clickable hyperlinks
\usepackage[bottom]{footmisc}       % Prevents the table going below the footnote
\usepackage{nccmath}      % Needed in the workaround for the ``aligncustom'' environment
\usepackage{amssymb}      % Used for black QED symbol
\usepackage{bm}           % Allows for bolding math symbols.
\usepackage{tabto}        % Allows to tab to certain point on a line
\usepackage{float}

\usepackage{mathtools} % for "\DeclarePairedDelimiter" macro
\DeclarePairedDelimiter{\floor}{\lfloor}{\rfloor}
\DeclarePairedDelimiter{\ceil}{\lceil}{\rceil}
\DeclarePairedDelimiter{\abs}{\lvert}{\rvert}
\DeclarePairedDelimiter{\norm}{\lVert}{\rVert}
\DeclarePairedDelimiter{\wnorm}{\lVert}{\rVert_{A}}

\newcommand{\red}[1]{{\color{red} #1}}
\newcommand{\blue}[1]{{\color{blue} #1}}

%---------------------------------------------------%
%     Define Distances Used for Document Margins    %
%---------------------------------------------------%
\newcommand{\hangindentdistance}{1cm}
\newcommand{\defaultleftmargin}{0.25in}
\newcommand{\questionleftmargin}{-.5in}
%\setlength{\parindent}{0pt}
\setlength{\parskip}{1em}
\setlength{\oddsidemargin}{\defaultleftmargin}




%---------------------------------------------------%
%      Configure the Document Header and Footer     %
%---------------------------------------------------%
% Set up page formatting
\usepackage{todonotes}
\usepackage{fancyhdr}       % Used for every page footer and title.
\pagestyle{fancy}
\fancyhf{}                  % Clears both the header and footer
\renewcommand{\headrulewidth}{0pt}    % Eliminates line at the top of the page.
\fancyfoot[LO]{\course\ -- \assnName}  % Left
\fancyfoot[CO]{\thepage}    % Center
\fancyfoot[RO]{\name}       %Right


%---------------------------------------------------%
%           Define the Title Page Entries           %
%---------------------------------------------------%
\title{\textbf{\course\ -- \assnName}}
\author{\name}


%---------------------------------------------------%
% Define the Environments for the Problem Inclusion %
%---------------------------------------------------%
\usepackage{scrextend}
\newcounter{subProbCount}     % Initialize the subproblem counter
\newcounter{problemCount}
\setcounter{problemCount}{0}  % Reset the subproblem counter
\newenvironment{problemshell}{
  \begin{addmargin}[\questionleftmargin]{0em}
    \par%
    \medskip
    \leftskip=0pt\rightskip=0pt%
    \setlength{\parindent}{0pt}
    \bfseries
}
{
    \par\medskip
  \end{addmargin}
}
\newenvironment{problem}
{%
  \refstepcounter{problemCount} % Increment the subproblem counter.  This must be before the exercise to ensure proper numbering of claims and lemmas.
  % Ref in the step counter allows the problem number to be labelled
  \begin{problemshell}
    \noindent \textit{Exercise~\#\arabic{problemCount}} \\
  }
  {
  \end{problemshell}
  \setcounter{subProbCount}{1} % Reset the subproblem counter
}
\newenvironment{subproblem}
{%
  \begin{problemshell}
    \setlength{\leftskip}{\hangindentdistance}
    % Print the subproblem count and offset to the left
    \hspace{-\hangindentdistance}(\alph{subProbCount}) \tabto{0pt}
}
{
    \stepcounter{subProbCount} % Increment the subproblem counter
  \end{problemshell}
}

% Change interline spacing.
\renewcommand{\baselinestretch}{1.1}
\newenvironment{aligncustom}
{ \csname align*\endcsname % Need to do this instead of \begin{align*} because of LaTeX bug.
  \centering
}
{
  \csname endalign*\endcsname
}


\newcommand{\transpose}{^{\textnormal{T}}}
\newcommand{\plotDim}{8cm}
\newcommand{\xopt}{x^{*}}


\begin{document}
  \maketitle

  \noindent
  \textbf{Name}: Zayd Hammoudeh \\
  \textbf{Course Name}: \course \\
  \textbf{Assignment Name}: \assnName \\
  \textbf{Due Date}: May 4, 2018 \\
  \textbf{Student Discussions}: I discussed the problems with the following students.  All write-ups were prepared separately and independently. \\
  \vspace{-2.5em}
  \begin{itemize}
    \item Ben Sherman
    \item Bernardo Torres
  \end{itemize}

  \newpage
  \begin{problem}
  Exercise 2.1 in Nocedal and Wright.

  Compute the gradient, $\nabla f(x)$, and Hessian, $\nabla^{2}f(x)$ of the Rosenbrock function

  \[ f(x) = 100(x_{2} - x_{1}^{2})^{2} + (1-x_{1})^{2} \textrm{.} \]
\end{problem}

\noindent
The gradient $\nabla f(x)$ is:

\begin{aligncustom}
  \nabla f(x) &=  \begin{bmatrix}
                    \frac{\partial f(x)}{\partial x_1} \\
                    \frac{\partial f(x)}{\partial x_2}
                  \end{bmatrix} \\
              &=  \begin{bmatrix}
                    -400 x_{1} (x_{2} - x_{1}^{2}) - 2(1-x_{1}) \\
                    200 (x_{2} - x_{1}^{2})
                  \end{bmatrix}\\
              &=  \boxed{
                    \begin{bmatrix}
                      400 x_{1}^3 - 400x_{1}x_{2} +2x_1 -2 \\
                      200 x_{2} - 200 x_{1}^{2}
                    \end{bmatrix}
                  } \textrm{.} 
\end{aligncustom}

\noindent
The Hessian~$\nabla^2 f(x)$ equals:                

\begin{aligncustom}
  \nabla^{2} f(x) &=  \begin{bmatrix}
                        \frac{\partial^{2} f(x)}{\partial x_1^{2}} & \frac{\partial^{2} f(x)}{\partial x_1 \partial x_2} \\
                        \frac{\partial^{2} f(x)}{\partial x_2 \partial x_1} & \frac{\partial^{2} f(x)}{\partial x_{2}^{2}}
                      \end{bmatrix} \\
                &=  \boxed{
                      \begin{bmatrix}
                        1200x_{1}^{2} -400x_2 + 2 & -400 x_{1} \\
                        -400x_{1} & 200
                      \end{bmatrix}
                    }
\end{aligncustom}



\begin{subproblem}
  Show that $x^{*} = (1,1)\transpose$ is the only local minimizer of the function, and that the Hessian at that point is positive definite.
\end{subproblem}

To determine the minimizer(s) (if any), set the gradient equal to~0.  Therefore,

\[  \begin{bmatrix}
      400 x_{1}^3 - 400x_{1}x_{2} +2x_1 -2 \\
      200 x_{2} - 200 x_{1}^{2}
    \end{bmatrix}  
    = \begin{bmatrix}
        0 \\ 0
      \end{bmatrix} \textrm{.} \]

\noindent
Therefore, $x_2 = x_{1}^{2}$.  If we substitute this into the first equation, we get:

\begin{aligncustom}
  400 x_{1}^3 - 400x_{1}x_{2} +2x_1 -2 &= 0 \\
  400 x_{1}^3 - 400x_{1}x_{1}^2 +2x_1 -2 &= 0 \\
  2x_1 -2 &= 0 \\
  x_1 = 1
\end{aligncustom}

\noindent
It is clear then that the only root is $\boxed{(1,1)\transpose}$.

The Hessian matrix~$\nabla^{2}f(x^{*})$ of the Rosenbrock function equals:

\[
\nabla^{2}f(x^{*}) =  \begin{bmatrix}
                        802 & -400 \\
                        -400 & 200
                      \end{bmatrix} \textrm{.}\]

\noindent
Hessian matrix,~$\nabla^{2}f(x^{*})$, is positive definite (PD) if all of its eigenvalues are positive.  These are found via:

\begin{aligncustom}
  \abs{A-\lambda I}  =  \begin{vmatrix}
                          802 - \lambda  & -400 \\
                          -400           & 200 - \lambda
                        \end{vmatrix}
                    &=  0 \\
  160400 - 1002 \lambda + \lambda^2 - 160000 &= 0 \\
  400- 1002 \lambda + \lambda^2 &= 0
\end{aligncustom}

\noindent
Solving for $\lambda$ (in Matlab), the eigenvalues are $\boxed{\lambda_1\approx 0.3994}$ and $\boxed{\lambda_2 \approx 1001.6}$.  Therefore, $\nabla^{2}f(x^{*})$ is PD.


  \newpage
  \begin{problem}\label{prob:02}
  Code Algorithm~4.1 in Nocedal and Wrihght with:
  
  \begin{enumerate}
    \item Cauchy point method for the subproblem
    \item Dog-leg method based on the results for Exercise~\ref{prob:01}.
  \end{enumerate}

  \noindent
  Test and compare the performance of the methods on the following problem:
  
  \[\min_{x\in \mathbb{R}^n} f(x) = \log\left(1+x\transpose Qx\right)\]
  
  \noindent
  where $Q$ is symmetric and positive definite matrix.
\end{problem}


The gradient,~$g$, of $f$ is defined as: 

\[g = f'(x) = \frac{2Qx}{1+x\transpose Q x}\text{.}\]

\noindent
Table~\ref{tab:p02:experimentParameters} defines the parameters used in the experiments.  Note that $\mathcal{U}(a,b)$ represents a uniform random variable selected from the range $[a,b)$. Since $Q$ is positive, definite, then for all $x \ne [0]^n \implies x\transpose Q x > 0$.  Therefore, since $\log$ is a monotonically increasing function, $f$ is minimized when $\xopt = [0]^n$.

\begin{table}[h]
  \caption{Parameters used in the experiments for Exercise~\#\ref{prob:02}}\label{tab:p02:experimentParameters}
  \centering
  \begin{tabular}{|c|c|}
    \hline
    \textbf{Name} & \textbf{Value} \\\hline
    \hline
    $\log$    &   Base 10\\\hline
    $n$       &   100\\\hline
    $\lambda$ &   $\mathcal{U}(10,1000)$ \\\hline
    $x_0$     &   Random vector from $\{[0,1)\}^{n}$\\\hline
    $\xopt$   &   $[0]^n$\\\hline
  \end{tabular}
\end{table}





  \newpage
  \begin{problem}\label{prob:03}%%
  Code Algorithm 7.5, and test it on the extended Rosenbrock function
  
  \[ f(x) = \sum_{i=1}^{n/2} \left[ \alpha(x_{2i} - x^2_{2i-1})^2 + (1-x_{2i-1})^2 \right] \text{.} \]
  
  where $\alpha$ is a parameter that you can vary (for example, 1 or 100). The solution is $\xopt=(1,1,...,1)\transpose$, $f^{*}=0$. Choose the starting point as $(-1,-1,...,-1)\transpose$. Observe the behavior of your program for various values of the memory parameter $m$.
\end{problem}

For even valued $n$, the gradient of $f$ is:

\[
\nabla f(x) = \left\{
                \begin{array}{cl}
                  -4\alpha x_{j}(x_{j+1} - x^2_{j}) - 2x_{j}(1-x_{j}) & j \text{ is odd}\\
                  2\alpha(x_j - x^{2}_{j-1}) & j \text{ is even}
                \end{array}
              \right.
\]

\noindent
where ${j \in \{1,\ldots,n\}}$.  Since LBFGS relies on line search, $\phi(\alpha) = f(x_k + \alpha_k p_k)$.  In addition, by the chain rule, 

\begin{aligncustom}
  \phi'(\alpha) &= \frac{\partial f(x_k+\alpha p_k)}{\partial \alpha} \\
                &= \frac{\partial f(x_k+\alpha p_k)}{\partial x_k} \cdot \frac{\partial x_k}{\partial \alpha} \\
                &= \nabla f(x_k + \alpha p_k) \cdot p_k \text{.}
\end{aligncustom}




  \newpage
  \begin{problem}
  Consider the problem of minimizing

  \[f(x_{1},x_{2},\cdots,x_{n}) = \sum_{i=1}^{n-1}[100(x^{2}_{i} - x_{i+1})^{2} +(x_{i} - 1)^{2}]\textrm{.}\]

  The global minimum is at $x^{∗} = [1, 1,\cdots,1]$. Numerically solve this problem using nonlinear conjugate gradient algorithms:
  \begin{enumerate}
    \item FR (Algorithm 5.4)
    \item FR with restart based on (5.52)
    \item PR based on (5.44)
  \end{enumerate}

  \noindent
  and compare their performance. (In the numerical experiments you can set ${n = 100}$ or any number that is not too small. The initial condition can be chosen as a random vector, for example, $2∗\text{rand}(n,1)$.)
\end{problem}

As their names indicate, the three non-linear conjugate gradient methods rely on the calculation of the gradient.  The gradient of $f$ is:

\[ \nabla f = \left\{
                \begin{array}{lc}
                  400x_i(x_{i}^{2} - x_{i+1}) + 2(x_i - 1)\text{,} & i = 1 \\
                  400x_i(x_{i}^{2} - x_{i+1}) + 2(x_i - 1) -200(x^{2}_{i-1} - x_{i})\text{,} & 1 < i < n \\
                  -200(x^{2}_{i-1} - x_{i})\text{,} & i = n \\
                \end{array}
              \right. \text{.} \]
              
\noindent
For the inexact line search, define $\psi_i = x_i + \alpha p_i$.  Therefore, the derivative of $\phi$ is

\[ \phi' =  \sum_{i=1}^{n-1} 400 p_i \psi_i (\psi_{i}^{2} - \psi_{i+1}) 
            + 2p_i(\psi_i - 1) \text{.} \]

\noindent
The parameters used for this experiment are specified in Table~\ref{tab:p04:ExperimentParams}.

\begin{table}[h]
  \centering
  \begin{tabular}{|c|c|}
    \hline
    \textbf{Parameter} & \textbf{Value} \\
    \hline\hline
    $n$     & 100 \\\hline
    $x_{0}$ & $2\cdot\text{rand}(n,1)$\\\hline
  \end{tabular}
  \caption{Experiment parameters for problem~\#4}\label{tab:p04:ExperimentParams}
\end{table}

Figures~\ref{fig:p04:FR},~\ref{fig:p04:FRwithRestart}, and~\ref{fig:p04:PR} show the performance of Fletcher-Reeves, Fletcher-Reeves with Restart, and Polak-Ribi\'{e}re respectively.


\newpage
\begin{figure}
  \centering
  \begin{subfigure}[t]{0.48\textwidth}
    \begin{tikzpicture}
  \pgfplotstableread[col sep=comma] {pgfplots/plot_data/p04_fr.csv}\thedata
  \begin{axis}[
      width=\plotDim,
      height=\plotDim,
      xmin=0,
      xmax=2000,
      minor x tick num=4,
      ymin=1.8,
      ymax=4,
      every tick label/.append style={font=\scriptsize},  % Reduce axis font size
      xlabel={Iteration \#},
      ylabel={$\log_{10} (f(x) - f(\xopt))$},
      xlabel style={font=\scriptsize},
      ylabel style={font=\scriptsize},
      ylabel shift = -8pt,
      title={Fletcher-Reeves},
    ]
    \addplot[
%      mark size=0.8pt,
      color=blue,
%      mark=*,
      mark options={blue,fill=blue}
    ]
    table[x index=0,y index=1] {\thedata};
  \end{axis}
\end{tikzpicture}

    \caption{}\label{fig:p04:FR}
  \end{subfigure}
  ~
  \begin{subfigure}[t]{0.48\textwidth}
    \begin{tikzpicture}
  \pgfplotstableread[col sep=comma] {pgfplots/plot_data/p04_fr_with_restart.csv}\thedata
  \begin{axis}[
      width=\plotDim,
      height=\plotDim,
      ymin=-15,
      ymax=5,
      minor y tick num=4,
      xmin=0,
      xmax=2000,
      minor x tick num=4,
      every tick label/.append style={font=\scriptsize},  % Reduce axis font size
      xlabel={Iteration \#},
      ylabel={$\log_{10} (f(x) - f(\xopt))$},
      xlabel style={font=\scriptsize},
      ylabel style={font=\scriptsize},
      ylabel shift = -8pt,
      title={Fletcher-Reeves Method with Restart},
    ]
    \addplot[
%      mark size=0.8pt,
      color=blue,
%      mark=*,
      mark options={blue,fill=blue}
    ]
    table[x index=0,y index=1] {\thedata};
  \end{axis}
\end{tikzpicture}

    \caption{}\label{fig:p04:FRwithRestart}
  \end{subfigure}
  \caption{Problem~\#4 performance for Fletcher-Reeves without and with restart}
\end{figure}

\begin{figure}
  \centering
  \begin{tikzpicture}
  \pgfplotstableread[col sep=comma] {pgfplots/plot_data/p04_pr.csv}\thedata
  \begin{axis}[
      width=\plotDim,
      height=\plotDim,
      xmin=0,
      xmax=400,
      minor x tick num=3,
      ymin=-16,
      ymax=5,
      minor y tick num=4,
      every tick label/.append style={font=\scriptsize},  % Reduce axis font size
      xlabel={Iteration \#},
      ylabel={$\log_{10} f(x)$},
      xlabel style={font=\scriptsize},
      ylabel style={font=\scriptsize},
      ylabel shift = -4pt,
      title={Polak-Ribi\`{e}re},
    ]
    \addplot[
%      mark size=0.8pt,
      color=blue,
%      mark=*,
      mark options={blue,fill=blue}
    ]
    table[x index=0,y index=1] {\thedata};
  \end{axis}
\end{tikzpicture}

  \caption{Problem~\#4 performance for the Polak-Ribi\'{e}re Method}\label{fig:p04:PR}
\end{figure}


  \newpage
  \vspace*{\fill}
  \centering
  \noindent
  \textbf{\Huge{Problem \#3 \\~\\ Source Code}}
  \vspace*{\fill}

  \newpage
  \vspace*{\fill}
  \centering
  \noindent
  \textbf{\Huge{Problem \#4 \\~\\ Source Code}}
  \vspace*{\fill}
\end{document}

