\documentclass{report}


\newcommand{\name}{Zayd Hammoudeh}
\newcommand{\course}{AMS230}
\newcommand{\assnName}{Homework~\#2}

\usepackage{fullpage}
\usepackage[skip=4pt]{caption}      % ``skip'' sets the spacing between the figure and the caption.
\usepackage{tikz}
\usepackage{pgfplots}     % Needed for plotting
\pgfplotsset{compat=newest}
\usepackage{subcaption}

\usepackage{amsmath}      % Allows for piecewise functions using the ``cases'' construct
\usepackage{amsthm}
\usepackage{siunitx}      % Allows for ``S'' alignment in table to align by decimal point
\usepackage{enumitem}     % Reference enumerated item lists.

\usepackage[obeyspaces,spaces]{url} % Used for typesetting with the ``path'' command
\usepackage[hidelinks]{hyperref}    % Make the cross references clickable hyperlinks
\usepackage[bottom]{footmisc}       % Prevents the table going below the footnote
\usepackage{nccmath}      % Needed in the workaround for the ``aligncustom'' environment
\usepackage{amssymb}      % Used for black QED symbol
\usepackage{bm}           % Allows for bolding math symbols.
\usepackage{tabto}        % Allows to tab to certain point on a line
\usepackage{float}

\usepackage{mathtools} % for "\DeclarePairedDelimiter" macro
\DeclarePairedDelimiter{\floor}{\lfloor}{\rfloor}
\DeclarePairedDelimiter{\ceil}{\lceil}{\rceil}
\DeclarePairedDelimiter{\abs}{\lvert}{\rvert}
\DeclarePairedDelimiter{\norm}{\lVert}{\rVert}
\DeclarePairedDelimiter{\wnorm}{\lVert}{\rVert_{A}}

\newcommand{\red}[1]{{\color{red} #1}}
\newcommand{\blue}[1]{{\color{blue} #1}}

%---------------------------------------------------%
%     Define Distances Used for Document Margins    %
%---------------------------------------------------%
\newcommand{\hangindentdistance}{1cm}
\newcommand{\defaultleftmargin}{0.25in}
\newcommand{\questionleftmargin}{-.5in}
%\setlength{\parindent}{0pt}
\setlength{\parskip}{1em}
\setlength{\oddsidemargin}{\defaultleftmargin}




%---------------------------------------------------%
%      Configure the Document Header and Footer     %
%---------------------------------------------------%
% Set up page formatting
\usepackage{todonotes}
\usepackage{fancyhdr}       % Used for every page footer and title.
\pagestyle{fancy}
\fancyhf{}                  % Clears both the header and footer
\renewcommand{\headrulewidth}{0pt}    % Eliminates line at the top of the page.
\fancyfoot[LO]{\course\ -- \assnName}  % Left
\fancyfoot[CO]{\thepage}    % Center
\fancyfoot[RO]{\name}       %Right


%---------------------------------------------------%
%           Define the Title Page Entries           %
%---------------------------------------------------%
\title{\textbf{\course\ -- \assnName}}
\author{\name}


%---------------------------------------------------%
% Define the Environments for the Problem Inclusion %
%---------------------------------------------------%
\usepackage{scrextend}
\newcounter{subProbCount}     % Initialize the subproblem counter
\newcounter{problemCount}
\setcounter{problemCount}{0}  % Reset the subproblem counter
\newenvironment{problemshell}{
  \begin{addmargin}[\questionleftmargin]{0em}
    \par%
    \medskip
    \leftskip=0pt\rightskip=0pt%
    \setlength{\parindent}{0pt}
    \bfseries
}
{
    \par\medskip
  \end{addmargin}
}
\newenvironment{problem}
{%
  \refstepcounter{problemCount} % Increment the subproblem counter.  This must be before the exercise to ensure proper numbering of claims and lemmas.
  % Ref in the step counter allows the problem number to be labelled
  \begin{problemshell}
    \noindent \textit{Exercise~\#\arabic{problemCount}} \\
  }
  {
  \end{problemshell}
  \setcounter{subProbCount}{1} % Reset the subproblem counter
}
\newenvironment{subproblem}
{%
  \begin{problemshell}
    \setlength{\leftskip}{\hangindentdistance}
    % Print the subproblem count and offset to the left
    \hspace{-\hangindentdistance}(\alph{subProbCount}) \tabto{0pt}
}
{
    \stepcounter{subProbCount} % Increment the subproblem counter
  \end{problemshell}
}

% Change interline spacing.
\renewcommand{\baselinestretch}{1.1}
\newenvironment{aligncustom}
{ \csname align*\endcsname % Need to do this instead of \begin{align*} because of LaTeX bug.
  \centering
}
{
  \csname endalign*\endcsname
}


\newcommand{\transpose}{^{\textnormal{T}}}
\newcommand{\plotDim}{8cm}
\newcommand{\xopt}{x^{*}}


\begin{document}
  \maketitle

  \noindent
  \textbf{Name}: Zayd Hammoudeh \\
  \textbf{Course Name}: \course \\
  \textbf{Assignment Name}: \assnName \\
  \textbf{Due Date}: May 4, 2018 \\
  \textbf{Student Discussions}: I discussed the problems with the following students.  All write-ups were prepared separately and independently. \\
  \vspace{-2.5em}
  \begin{itemize}
    \item Ben Sherman
    \item Bernardo Torres
  \end{itemize}

  \newpage
  \begin{problem}\label{prob:01}
  Consider the dog-leg path:
  
  \[
    \tilde{p}(\tau) = \left\{
                        \begin{array}{cl}
                          \tau p^{U}, & 0\leq \tau \leq 1\\
                          p^{U} + (\tau - 1)(p^{B} - p^{U}), & 1 \leq \tau \leq 2
                        \end{array}
                      \right.
  \]
  
  \noindent
  where 
  
  \begin{aligncustom}
    p^{U} &= -\frac{\norm{g}^2}{g\transpose B g} g\\
    B \cdot p^{B} &= -g \text{.}
  \end{aligncustom}

  \noindent
  Support that symmetric matrix $B$ and vector $g$ satisfy:
  
  \begin{enumerate}
    \item $g\transpose B g > 0$
    \item $\left(p^{U}\right)\transpose \left(p^{B} - p^{U}\right) > 0$
  \end{enumerate}

  \noindent
  Prove that:

  \begin{enumerate}[i.]
    \item\label{item:p01:prop1} $\norm{\tilde{p}(\tau)}$ is an increasing function of $\tau$
    \item\label{item:p01:prop2} $m(\tilde{p}(\tau))$ is a decreasing function of $\tau$
  \end{enumerate}

  \noindent
  where $m(p) = g\transpose p + \frac{1}{2}p\transpose B p$.
\end{problem}

\begin{proof}
  Based off Lemma~4.2 from Nocedal and Wright.  There are two cases based on the value of $\tau$.
  
  \noindent
  \textit{Case~\#1}: $0 \leq \tau \leq 1$
  
  \noindent
  Property~\ref{item:p01:prop1}: $\norm{\tilde{p}(\tau)} = \tau \norm{p^U}$.  This is clearly increasing for $\tau \in [0,1]$ since $\norm{p^U}$ is strictly positive.
  
  \noindent
  Property~\ref{item:p01:prop2}: This can be simplified via:
  
  \begin{aligncustom}
    m(\tilde{p}(\tau)) &= \tau g\transpose p^{U} + \frac{\tau^{2}}{2} \left(p^{U}\right)\transpose B p^{U} \\
                       &= -\tau \frac{\left(\norm{g}^2\right)^{2}}{g\transpose B g} + \frac{\tau^{2}}{2}\frac{\left(\norm{g}^2\right)^{2}}{g\transpose B g}  \\
                       &= \left(-\tau + \frac{\tau^2}{2}\right)\frac{\left(\norm{g}^2\right)^{2}}{g\transpose B g}
  \end{aligncustom}

  \noindent
  Both $\norm{g}^{2}$ and $g\transpose B g$ are strictly positive.  For $\tau \in [0,1]$, $-\tau + \tau^2/2$ is strictly decreasing.
  
  \noindent
  \textit{Case~\#2}: $1 \leq \tau \leq 2$\\
  
  \noindent
  Property~\ref{item:p01:prop1}: Define $h(\alpha)$ where $\alpha \in (0,1)$ as:
  
  \begin{aligncustom}
    h(\alpha) &= \frac{1}{2}\norm{\tilde{p}(1+\alpha)}^2\\
              &= \frac{1}{2} \norm{p^{U} +\alpha(p^{B}-p^{U})} \\
              &= \frac{1}{2}\norm{p^U}^2 + \alpha\left(p^{U}\right)\transpose(p^{B}-p^{U}) + \frac{1}{2}\alpha^2\norm{p^{B}-p^{U}}^2
  \end{aligncustom}

  \noindent
  To prove the property, it is sufficient to show that $h'(\alpha) \geq 0$ for $\alpha \in (0,1)$.  Therefore,
  
  \begin{aligncustom}
    h'(\alpha)  &= -\left(p^{U}\right)\transpose \left(p^{U} - p^{B} \right) + \alpha\norm{p^{U} - p^{B}}^{2}\\
                &\geq -(p^{U})\transpose (p^{U}-p^{B}) \\
                &= \left(\frac{g\transpose g}{g\transpose B g}\right)g\transpose\left(-\left(\frac{g\transpose g}{g\transpose B g}\right)g + B^{-1}g \right) \\
                &= g\transpose g \frac{g\transpose B^{-1} g}{g\transpose B g}\left[ 1 - \frac{(g\transpose g)^{2}}{\left(g\transpose B g\right) \left(g\transpose B^{-1} g \right)  } \right] \\
                &\geq 0
  \end{aligncustom}

  \noindent
  by the Cauchy-Schwarz inequality.

  \noindent
  Property~\ref{item:p01:prop2}: Define  $\hat{h}(\alpha)=m(\tilde{p}(1+\alpha))$.  If $h'(\alpha) \leq 0$ for $\alpha \in (0,1)$ then the property holds.  Using the definition of $\tilde{p}(\tau)$ in the exercise description and the definition of trust region, we find:
  
  \begin{aligncustom}
    \hat{h}'(\alpha)  &= (p^{B} - p^{U})\transpose (g + Bp^{U}) \\
                      &\leq (p^{B} - p^{U})\transpose (g+Bp^{U} + B(p^{B} - p^{U})) \\
                      &= (p^{B} - p^{U})\transpose (g + Bp^{B}) = 0
  \end{aligncustom}

  \noindent
  given $B \cdot p^{B} = -g$.
\end{proof}


  \newpage
  \begin{problem}\label{prob:02}%%
  Code Algorithm~4.1 in Nocedal and Wrihght with:
  
  \begin{enumerate}
    \item Cauchy point method for the subproblem
    \item Dog-leg method based on the results for Exercise~\ref{prob:01}.
  \end{enumerate}

  \noindent
  Test and compare the performance of the methods on the following problem:
  
  \[\min_{x\in \mathbb{R}^n} f(x) = \log\left(1+x\transpose Qx\right)\]
  
  \noindent
  where $Q$ is symmetric and positive definite matrix.
\end{problem}


The gradient,~$g$, of $f$ is defined as: 

\[g = f'(x) = \frac{2Qx}{1+x\transpose Q x}\text{.}\]

\noindent
Table~\ref{tab:p02:experimentParameters} defines the parameters used in the experiments.  $Q$ is constructed in the same way it was constructed in homework~\#2.  Note that $\mathcal{U}(a,b)$ represents a uniform random variable selected from the range $[a,b)$. Since $Q$ is positive, definite, then for all $x \ne [0]^n \implies x\transpose Q x > 0$.  Therefore, since $\log$ is a monotonically increasing function, $f$ is minimized when $\xopt = [0]^n$.

\begin{table}[h]
  \caption{Parameters used in the experiments for Exercise~\#\ref{prob:02}}\label{tab:p02:experimentParameters}
  \centering
  \begin{tabular}{|c|c|}
    \hline
    \textbf{Name} & \textbf{Value} \\\hline
    \hline
    $\log$      &   $\log_{10}$\\\hline
    $n$         &   100\\\hline
    $\lambda$   &   $\mathcal{U}(10,1000)$ \\\hline
    $\Delta_0$  &   1 \\\hline
    $x_0$       &   Random vector from $\{[0,1)\}^{n}$\\\hline
    $\xopt$     &   $[0]^n$\\\hline
  \end{tabular}
\end{table}





  \newpage
  \begin{problem}\label{prob:03}%%
  Code Algorithm 7.5, and test it on the extended Rosenbrock function
  
  \[ f(x) = \sum_{i=1}^{n/2} \left[ \alpha(x_{2i} - x^2_{2i-1})^2 + (1-x_{2i-1})^2 \right] \text{.} \]
  
  where $\alpha$ is a parameter that you can vary (for example, 1 or 100). The solution is $\xopt=(1,1,...,1)\transpose$, $f^{*}=0$. Choose the starting point as $(-1,-1,...,-1)\transpose$. Observe the behavior of your program for various values of the memory parameter $m$.
\end{problem}

For even valued $n$, the gradient of $f$ is:

\[
\nabla f(x) = \left\{
                \begin{array}{cl}
                  -4\alpha x_{j}(x_{j+1} - x^2_{j}) - 2x_{j}(1-x_{j}) & j \text{ is odd}\\
                  2\alpha(x_j - x^{2}_{j-1}) & j \text{ is even}
                \end{array}
              \right.
\]

\noindent
where ${j \in \{1,\ldots,n\}}$.  Since LBFGS relies on line search, $\phi(\alpha) = f(x_k + \alpha_k p_k)$.  In addition, by the chain rule, 

\begin{aligncustom}
  \phi'(\alpha) &= \frac{\partial f(x_k+\alpha p_k)}{\partial \alpha} \\
                &= \frac{\partial f(x_k+\alpha p_k)}{\partial x_k} \cdot \frac{\partial x_k}{\partial \alpha} \\
                &= \nabla f(x_k + \alpha p_k) \cdot p_k \text{.}
\end{aligncustom}

Table~\ref{tab:p03:ExperimentParameters} lists the experiment parameters for this problem.  The implementation of line search remains essentially unchanged from that used in homeworks~\#1 and~\#2.  When $\alpha$ in the extended Rosenbrock function was set to~1, the model converged too quickly.  As such, $\alpha$ was set to~100.

\begin{table}[h]
  \centering
  \caption{Experiment parameters for problem~\#3}\label{tab:p03:ExperimentParameters}
  \begin{tabular}{|c|c|}
    \hline
    \textbf{Name} & \textbf{Value} \\\hline\hline
    $n$         & 1,000 \\\hline
    $\alpha$    & 100 \\\hline
    $x_0$       & $[-1]^{n}$ \\\hline
    $\alpha_0$  & 0 \\\hline
    $c_1$       & 0.1 \\\hline
    $c_2$       & 0.45 \\\hline
  \end{tabular}
\end{table}

When $m=0$, i.e., the algorithm behaves in a memoryless fashion and converges as traditional line search, it took 2,340 iterations to converge (not shown).  Figure~\ref{fig:p03:ResultsPlot} shows the results for different values of $m$.  As expected, when $m=1$, it took the longest to converge.  For $m\geq 2$, the convergence rate was essentially the same.  While for $m=2$ and $m=5$, the convergence was marginally, the difference was marginal and not unexpected as excluding some gradients may cause the algorithm to perform better in some cases.  For $m=10$, $m=20$, and maximum $m$ (i.e., save all previous results), the algorithm performed essentially the same.  This means that only the most recent gradients affected the results and memory of distant iterations does not improve the results for this function.

\begin{figure}
 \centering
%\tikzstyle{dotted}=                  [dash pattern=on \pgflinewidth off 2pt]
%\tikzstyle{densely dotted}=          [dash pattern=on \pgflinewidth off 1pt]
%\tikzstyle{loosely dotted}=          [dash pattern=on \pgflinewidth off 4pt]
%\tikzstyle{dashed}=                  [dash pattern=on 3pt off 3pt]
%\tikzstyle{densely dashed}=          [dash pattern=on 3pt off 2pt]
%\tikzstyle{loosely dashed}=          [dash pattern=on 3pt off 6pt]
%\tikzstyle{dashdotted}=              [dash pattern=on 3pt off 2pt on \the\pgflinewidth off 2pt]
%\tikzstyle{densely dashdotted}=      [dash pattern=on 3pt off 1pt on \the\pgflinewidth off 1pt]
%\tikzstyle{loosely dashdotted}=      [dash pattern=on 3pt off 4pt on \the\pgflinewidth off 4pt]
\begin{tikzpicture}[
    % Dashed style is shown here: https://tex.stackexchange.com/questions/45275/tikz-get-values-for-predefined-dash-patterns
    dashdotted/.style         = {dash pattern=on 3pt off 2pt on \the\pgflinewidth off 2pt},
    loosely dashed/.style     = {dash pattern=on 3pt off 6pt},
    densely dashdotted/.style = {dash pattern=on 3pt off 1pt on \the\pgflinewidth off 1pt},
    densely dotted/.style     = {dash pattern=on \pgflinewidth off 1pt},
  ]
  \pgfplotstableread[col sep=comma] {pgfplots/plot_data/lbfgs_m=1.csv}\mOne
  \pgfplotstableread[col sep=comma] {pgfplots/plot_data/lbfgs_m=2.csv}\mTwo
  \pgfplotstableread[col sep=comma] {pgfplots/plot_data/lbfgs_m=5.csv}\mFive
  \pgfplotstableread[col sep=comma] {pgfplots/plot_data/lbfgs_m=10.csv}\mTen
  \pgfplotstableread[col sep=comma] {pgfplots/plot_data/lbfgs_m=20.csv}\mTwenty
  \pgfplotstableread[col sep=comma] {pgfplots/plot_data/lbfgs_max_mem.csv}\maxMem
  \begin{axis}[
      width=10cm,
      height=10cm,
      xmin=0,
      xmax=35,
      minor x tick num=3,
      ymin=-14,
      ymax=6,
      minor y tick num=1,
      every tick label/.append style={font=\scriptsize},  % Reduce axis font size
      xlabel={Iteration \#},
      ylabel={$\log_{10} f(x)$},
      xlabel style={font=\scriptsize},
      ylabel style={font=\scriptsize},
      ylabel shift = -4pt,
      title={L-BFGS performance comparison for different values of $m$},
    ]
    \addplot[
      color=red,
      thick
    ]
    table[x index=0,y index=1] {\mOne};
    \addplot[
      color=green,
      dashdotted,
      thick
    ]
    table[x index=0,y index=1] {\mTwo};
    \addplot[
      color=orange,
      loosely dashed,
      thick
    ]
    table[x index=0,y index=1] {\mFive};
    \addplot[
      color=blue,
      densely dashdotted,
      thick
    ]
    table[x index=0,y index=1] {\mTen};
    \addplot[
      color=purple,
      densely dotted,
      thick
    ]
    table[x index=0,y index=1] {\mTwenty};
    \addplot[
      color=black,
      dashed,
      thick
    ]
    table[x index=0,y index=1] {\maxMem};
    \legend{$m=1$, $m=2$, $m=5$, $m=10$, $m=20$, Max $m$};
  \end{axis}
\end{tikzpicture}

  \caption{Convergence of L-BFGS for different values of $m$ on the extended Rosenbrock function with $\alpha=100$}\label{fig:p03:ResultsPlot}
\end{figure}


  \newpage
  \begin{problem}
  Consider the steepest descent method with exact line searches applied to the convex quadratic function $f(x)=\frac{1}{2}x\transpose Q x - b\transpose x$, where $Q$ is symmetric and positive definite.  Show that the search direction at step $k+1$ is always orthogonal to the search direction at step~$k$, i.e.,~$p_{k}\transpose p_{k+1} = 0$ for all $k$.
\end{problem}

\noindent
In exact line search, the step size $\alpha$ is chosen such that:

\[ \alpha_k = \arg \min_{\alpha} f(x_k + \alpha p_k)\textrm{.} \]

\noindent
Therefore, taking the derivative of $f_{k+1}$ with respect to $\alpha$ yields:

\[ \frac{\partial f_{k+1}}{\partial \alpha_{k}} = 0 \textrm{.} \]

\noindent
By the chain rule, this becomes,

\[ \frac{\partial f_{k+1}}{\partial x_{k+1}} \cdot \frac{\partial x_{k+1}}{\partial \alpha_{k}} = 0 \textrm{.} \]

\noindent
Knowing that ${x_{k+1} = x_k + \alpha_{k} p_k}$, we can rearrange the terms to prove the required statement:

\begin{aligncustom}
  \nabla\transpose f_{k+1} p_k &= 0 \\
  p_{k+1}\transpose p_k &= 0 \\
  p_{k}\transpose p_{k+1} &= 0\textrm{.}
\end{aligncustom}



  \newpage
  \vspace*{\fill}
  \centering
  \noindent
  \textbf{\Huge{Problem \#3 \\~\\ Source Code}}
  \vspace*{\fill}

  \newpage
  \vspace*{\fill}
  \centering
  \noindent
  \textbf{\Huge{Problem \#4 \\~\\ Source Code}}
  \vspace*{\fill}
\end{document}

