\begin{problem}\label{prob:05}%%
  Exercise~12.18 in Nocedal and Wright.
  
  Consider the problem of finding the point on the parabola $y=\frac{1}{5}(x-1)^{2}$ that is closest to $(x,y)=(1,2)$ in the Euclidian norm sense.  We can formulate this problem as:
  
  \[\min f(x,y)=(x-1)^2  + (y-2)^2 ~~~~~~ \text{subject to } (x-1)^2 = 5y \text{.} \]
\end{problem}

The gradient of $f$ is:

\[\nabla f =  \begin{bmatrix}
                2(x-1)\\
                2(y-2)
              \end{bmatrix} \text{.}\]

\noindent
Rewrite $c_1(x,y)=(x-1)^2-5y=0$.  Its gradient is:

\[\nabla c_1 =  \begin{bmatrix}
                  2(x-1)\\
                  -5
                \end{bmatrix} \text{.}\]

\begin{subproblem}
  Find all the KKT points for this problem.  Is the LICQ satisfied?
\end{subproblem}

For a single constraint, Lagrangian multiplier $\lambda_1$ only holds when $\nabla f = \lambda_1 \nabla c_1$.  In the case of the first dimension, $\nabla f = \lambda_1 \nabla c_1$.  There are two possible cases:

\noindent
\textit{Case~\#1}: $\lambda=1$.  This means $2(y-2)=-5$ making $y=-1/2$.  Clearly, this not valid so we can drop this case.

\noindent
\text{Case~\#2}: $x=1$.  Trivially, this makes $y=0$ and $\lambda_1=4/5$.  Therefore, $(\xopt,y^{*})=(1,0)$.  Clearly then 

\[\nabla \mathcal{L}\left(0,1,\frac{4}{5}\right)=\nabla f(1,0)-\frac{4}{5} \nabla c_1(1,0) = \vec{0} \] 

\noindent
and $c_1(1,0)=0$ satisfying Eq.~(12.34a) and Eq.~(12.34b).  Since there is only an equality constraint, we can ignore KKT inequality constraints Eq.~(12.34c) and Eq.~(12.34d).  In addition, Eq.~(12.34e) follows from proving $c_1(\xopt)=0$.

Since there is only a single constraint, LICQ is established by showing only that $\nabla c_{1}(\xopt) = (0, -5) \ne \vec{0}$.  

\begin{subproblem}
  Which of these points are solutions?
\end{subproblem}

The point $(\xopt,y^{*})=(1,0)$ is a solution.

\begin{subproblem}
  By directly substituting the constraint into the objective function and eliminating the variable~$x$, we obtain an unconstrained optimization problem.  Show that the solutions of this problem cannot be solutions to the original problem.
\end{subproblem}

Substituting the constraint into the objective function yields:

\[ f(y)=5y + (y-2)^2 \text{.} \]

\noindent
Taking the derivative and solving for the root yields:

\begin{aligncustom}
  \frac{df}{dy} = 5 + 2(y-2) &=0 \\
                  2 (y-2)  &= -5 \\
                  y&= 2 - 5/2 \\
                  y&= -1/2 \text{.}
\end{aligncustom}

\noindent
Clearly this is an invalid root since the parabola is strictly positive.
