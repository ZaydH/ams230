\begin{problem}\label{prob:03}%%
  Exercise~12.14 in Nocedal and Wright (assume that $a$ is not a zero vector).

  Consider the half space defined by $H=\{x\in \mathbb{R}^n \vert a\transpose x + \alpha \geq 0 \}$ where $a\in \mathbb{R}^n$ and $\alpha \in \mathbb{R}$ are given.  Formulate and solve the optimization problem for finding the point $x$ in $H$ that has the smallest Euclidian norm.
\end{problem}

This can be framed as the constrained optimization problem below.

\[ \min f(x)=\sum_i x_i^2 \text{ s.t. } a\transpose x + \alpha \ge 0 \text{.}\]

\noindent
If $\alpha \ge 0$, then $f(x)$ is minimized at $\xopt =\vec{0}$ since $f(\vec{0})=0$.  For all other $x \ne \vec{0}$, $f(x)$ is strictly positive.   This also means $\lopt=0$.

Consider the more interesting case when $\alpha$ is less than~0.  First, the gradient of $f(x)$ equals:

\[\nabla f(x) = (2x_1,\ldots,2x_n) \text{.} \]

\noindent
Similarly, the gradient of the constraint $c(x)$ is:

\[\nabla c(x) = a \text{.}\]

\noindent
By the first KKT condition,

\begin{aligncustom}
  \mathcal{L}(\xopt,\lopt) = \nabla f(\xopt) - \lopt \nabla c(\xopt) &= 0\\
  2 \xopt - \lopt a = 0 \\
  \xopt = \frac{\lopt}{2}a \text{.}
\end{aligncustom}

\noindent
By the fifth KKT condition,

\begin{aligncustom}
  \lopt c(\xopt) &= 0\\
  \lopt (a\transpose \xopt +\alpha) &= 0
\end{aligncustom}

\noindent
$\lopt=0$ is not a valid value if $\alpha < 0$.  Instead continuing by the fourth KKT condition where $\lopt >0$,

\begin{aligncustom}
  a\transpose \xopt +\alpha &= 0 \\
  a\transpose \left(\frac{\lopt}{2}a\right) &= -\alpha \\
  \lopt &= -\frac{2\alpha}{\norm{a}^2}
\end{aligncustom}

\noindent
This makes

\[\boxed{\xopt = -\frac{\alpha}{\norm{a}^2}a} \text{.}\]

\noindent
Since $\alpha$ is negative, $\lopt$ is positive satisfying the fourth KKT condition.  It is also trivial to see the third and fifth KKT conditions hold for these values of $\xopt$ and $\lopt$.  The second KKT condition does not apply since there are no equality constraints.  LICQ also holds since $\nabla c=a\ne\vec{0}$.