\begin{problem}\label{prob:03}%%
  Exercise~12.14 in Nocedal and Wright (assume that $a$ is not a zero vector).
  
  Consider the half space defined by $H=\{x\in \mathbb{R}^n \vert a\transpose x + \alpha \geq 0 \}$ where $a\in \mathbb{R}^n$ and $\alpha \in \mathbb{R}$ are given.  Formulate and solve the optimization problem for finding the point $x$ in $H$ that has the smallest Euclidian norm.
\end{problem}

This can be framed as the constrained optimization problem below.

\[ \min f(x)=\sum_i x_i^2 \text{ s.t. } a\transpose x + \alpha \ge 0 \text{.}\]

\noindent
If $\alpha > 0$, then trivially the $\xopt$ that minimizes the $f(x)$ is the zero vector since $f(x)=0$.  For all other $x \ne \vec{0}$, $f(x)$ is strictly positive.  

Consider the more interesting case when $\alpha$ is less than~0.  First, the gradient of $f(x)$ equals:

\[\nabla f(x) = (2x_1,\ldots,2x_n) \text{.} \]

\noindent
Similarly, the gradient of the constraint $c(x)$: 

\[\nabla c(x) = a \text{.}\]

\noindent
By the first KKT condition, 

\begin{aligncustom}
  \mathcal{L}(\xopt,\lambda^{*}) = \nabla f(\xopt) -\lambda^{*} \nabla c(\xopt) &= 0\\
  2 \xopt - \lambda^{*} a = 0 \\
  \xopt = \frac{\lambda^{*}}{2}a \text{.}
\end{aligncustom}

\noindent
By the fifth KKT condition,

\begin{aligncustom}
  \lambda^{*}c(\xopt) &= 0\\
  \lambda^{*}(a\transpose \xopt +\alpha) &= 0
\end{aligncustom}

\noindent
$\lambda=0$ is not a valid value if $\alpha < 0$.  Instead continuing by the fourth KKT condition where $\lambda >0$,

\begin{aligncustom}
  a\transpose \xopt +\alpha &= 0 \\
  a\transpose \left(\frac{\lambda^{*}}{2}a\right) &= -\alpha \\
  \lambda^{*} &= -\frac{2\alpha}{\norm{a}^2}
\end{aligncustom}

\noindent
This makes 

\[\boxed{\xopt = -\frac{\alpha}{\norm{a}^2}a} \text{.}\]

Since $\alpha$ is negative, $\lambda^{*}$ is positive satisfying the third KKT condition.  We already proved above the fourth and fifth KKT conditions.